\documentclass[16pt]{examdesign}


\usepackage{multicol}
\usepackage{wrapfig}
\usepackage{graphics}
\usepackage{graphicx}
\usepackage{pgf,pgfarrows,pgfnodes,pgfautomata,pgfheaps,pgfshade}
\usepackage{amsmath}
\usepackage{pifont}
%\usepackage{gensymb}
\usepackage{textcomp}
%\SectionFont{\large\sffamily}
\Fullpages
\ContinuousNumbering
%\ShortKey
%\NoKey

\DefineAnswerWrapper{}{}
\NumberOfVersions{4}
%\IncludeFromFile{foobar.tex}

\usepackage[english]{babel}
\usepackage[utf8]{inputenc}
\setrandomseed{1373}
\usepackage[portuguese]{algorithm2e}


\addtolength{\textwidth}{80pt}
\addtolength{\hoffset}{-55pt}


\addtolength{\voffset}{-30pt}
\addtolength{\textheight}{50pt}





\setlength{\fboxsep}{0pt}%
\setlength{\fboxrule}{1pt}

\newcommand{\aspas}{``}

\hyphenation{usa-do}
\hyphenation{tem-pe-ra-tu-ra}
% \newenvironment{Figure}
%   {\par\medskip\noindent\minipage{\linewidth}}
%   {\endminipage\par\medskip}

%\setlength\columnsep{25pt}
%Static/final
%Erro no código
%Teorica
%O que imprime? 
%Visibilidade

\begin{document}


\begin{examtop}

      \pgfdeclaremask{cefetLogo}{cefetLogo}
      \pgfdeclareimage[mask=cefetLogo,height=2cm]{cefetLogo}{../../../img/cefetLogo.png}
      
    \begin{multicols}{2}[]
    \begin{minipage}{2.5cm}
      \pgfuseimage{cefetLogo}
    \end{minipage}
      \columnbreak
      \begin{minipage}{400px}
      \begin{center}\large{\textbf{Matematica Discreta - Prova 1}}\end{center}~\\
      Professor: Daniel Hasan Dalip~~~~~~~~Valor: 33 Pontos~~~~~Belo Horizonte, 12/09/2016\\~\\
      Nome do(a) Aluno(a): \_\_\_\_\_\_\_\_\_\_\_\_\_\_\_\_\_\_\_\_\_\_\_\_	\_\_\_\_\_\_\_\_\_\_\_\_\_\_\_\_\_\_\_\_\_\_\_\_\_\_\_\_\_\_\_\_\_\_\_\_\_\_\_\_\_\_\_\_\_\_\_\_\_\_\_\_\_\_\_~
      Nota:\_\_\_\_\_\_\_\_\_\_\_\\       
      \end{minipage}



    \end{multicols}
  \hrulefill
  \begin{multicols}{2}[]
  
    \begin{itemize}
    \item Prova individual e sem consulta.\vspace{-6pt}
    \item Não é permitido o uso de aparelhos eletrônicos  \vspace{-6pt}
    \item A interpretação da questão faz parte da avaliação  \vspace{-6pt}
    \item A prova pode ser feita a lápis ou caneta  \vspace{-6pt}
    \item Qualquer tentativa de cola/fraude a prova será zerada  \vspace{-6pt}
    \end{itemize}
  \end{multicols}

\def\arraystretch{1}
\end{examtop}


 
 
\begin{multiplechoice}[title={Questões de múltipla escolha (3 pontos cada) },rearrange=no, resetcounter=no,suppressprefix,examcolumns=2,keycolumns=2]
%\begin{center}
%    \textbf{Prova entregue sem preencher gabarito abaixo não será considerada.}\vspace{1em}
%  
%    \begin{tabular}{ | p{0.92cm} | p{0.92cm} | p{0.92cm} | p{0.92cm} |p{0.92cm} |  }
%     \hline
%      1 & 2 & 3 & 4  & 5\\ \hline
%      (a) &(a) &(a) &(a)&(a)  \\ \hline
%      (b) &(b) &(b) &(b)&(b) \\ \hline
%      (c) &(c) &(c) &(c)&(c) \\ \hline
%      (d) &(d) &(d) &(d)&(d) \\ \hline
%      (e) &(e) &(e) &(e)&(e) \\ \hline
%    \end{tabular} ~\\
%   %\begin{multicols}{2}[]
%\end{center}
%   %\begin{multicols}{2}[]

  \begin{question}
	Considere p: João é esperto, q: João é rico. Assinale a  \textbf{NEGAÇÃO} da frase "João\word{{ não}{}} é esperto \word{{e}{e}{ou}{ou}}\word{{}{ não}} é rico", utilizando 
	lógica proposicional: 

	
		\choice{$\neg p \land q$}
		\choice{$p \land \neg q$}
		\choice{\word{{$\neg p \land \neg q$}{$\neg p \land \neg q$}{$\neg p \lor \neg q$}{$\neg p \lor \neg q$}}}
		\choice{$p \lor \neg q$}
		\choice{$\neg p \lor q$}
		
		%\choice{\word{{\FD}{\V}  {\FD}{\FD}}}
		\key{~\\~\\ Resposta: \word{{d} {e} {b} {a}}}
  \end{question}
 \begin{block}
 Para as questoes  2 até 5 considere os predicados:  $D(x,y)$  o aluno $x$ está matriculado na disciplina $y$, $O(x)$: O aluno $x$ usa óculos. 
 O domínio de $x$ são todos os alunos e $y$ são todas as disciplinas.
    
    Assinale, para cada questão, a expressão usando predicados e quantificadores que correponde à frase em português apresentada no enunciado. 
	\begin{question}
	    \newcommand{\FE}{$\exists x (O(x) \to D(x,\text{Cálculo}))$}
	    \newcommand{\VE}{$\exists x (O(x) \land D(x,\text{Cálculo}))$}
	    \newcommand{\VA}{$\forall x (O(x) \to D(x,\text{Cálculo}))$}
	    \newcommand{\FA}{$\forall x (O(x) \land D(x,\text{Cálculo}))$}
	    
	    \word{{Existe pelo menos um aluno usando óculos matriculado na disciplina de Cálculo}
		  {Existe pelo menos um aluno usando óculos matriculado na disciplina de Cálculo}
		  {Todo aluno que usa óculos está matriculado na disciplina de Cálculo}
		  {Todo aluno que usa óculos está matriculado na disciplina de Cálculo}
		  }
	    \choice{\word{{$\exists x \exists y (O(x) \to D(x,y))$}
			   {$\exists x \exists y (O(x) \to D(x,y))$}
			   {$\forall x \exists y (O(x) \land D(x,y))$}
			   {$\forall x \exists y (O(x) \land D(x,y))$}
			   }
		}
	    \choice{\word{{\FE}{\VE}}}
	    \choice{\word{{\VE}{\FE}}}
	    \choice{\word{{\VA}{\FA}}}
	    \choice{\word{{\FA}{\VA}}}
	    
	    \key{~\\~\\ Resposta: \word{{c} {b} {d} {e}}}
	\end{question}
	\begin{question}
	    \newcommand{\FE}{$\exists x (O(x) \to D(x,\text{Cálculo}))$}
	    \newcommand{\VE}{$\exists x (O(x) \land D(x,\text{Cálculo}))$}
	    \newcommand{\VA}{$\forall x (O(x) \to D(x,\text{Cálculo}))$}
	    \newcommand{\FA}{$\forall x (O(x) \land D(x,\text{Cálculo}))$}
	    
	    \word{
		  {Todo aluno que usa óculos está matriculado na disciplina de Cálculo}
		  {Todo aluno que usa óculos está matriculado na disciplina de Cálculo}
		  {Existe pelo menos um aluno usando óculos matriculado na disciplina de Cálculo}
		  {Existe pelo menos um aluno usando óculos matriculado na disciplina de Cálculo}
		  }
	    \choice{\word{
			    {$\forall x \exists y (O(x) \land D(x,y))$}
			   {$\forall x \exists y (O(x) \land D(x,y))$}
			    {$\exists x \exists y (O(x) \to D(x,y))$}
			   {$\exists x \exists y (O(x) \to D(x,y))$}
			   }
		}
	    \choice{\word{{\FE}{\VE}}}
	    \choice{\word{{\VE}{\FE}}}
	    \choice{\word{{\VA}{\FA}}}
	    \choice{\word{{\FA}{\VA}}}
	    
	    \key{~\\~\\ Resposta: \word{{d} {e} {c} {b}}}
	\end{question}
    
        \begin{question}%OK
    \word{
    	  {Existe pelo menos um aluno matriculado em todas as disciplinas}
	  {Todos os alunos estão matriculados em alguma disciplina.}
	  {Existe  pelo menos uma disciplina em que todos os alunos estão matriculados }
	  {Todas as disciplinas possuem pelo menos um aluno matriculado}
	  }
	  \choice{$\exists y \forall x (D(x,y))$}
	  \choice{$\forall y \exists x (D(x,y))$}
	  \choice{$\exists x \forall y (D(x,y))$}
	  \choice{$\forall x \exists y (D(x,y))$}
	  \choice{$\exists x \exists y (D(x,y))$}
	  \key{~\\~\\ Resposta: \word{{c} {d} {a} {b}}}
   \end{question}
    \begin{question}%OK
    \word{
	  {Existe  pelo menos uma disciplina em que todos os alunos estão matriculados }
	  {Todas as disciplinas possuem pelo menos um aluno matriculado}
	  {Existe pelo menos um aluno matriculado em todas as disciplinas}
	  {Todos os alunos estão matriculados em alguma disciplina.}
	  }
	  \choice{$\exists y \forall x (D(x,y))$}
	  \choice{$\forall y \exists x (D(x,y))$}
	  \choice{$\exists x \forall y (D(x,y))$}
	  \choice{$\forall x \exists y (D(x,y))$}
	  \choice{$\exists x \exists y (D(x,y))$}
	  \key{~\\~\\ Resposta: \word{{a} {b}{c} {d} }}
   \end{question}
\newpage
\end{block}


\end{multiplechoice}
\begin{fillin}[title={},
                    rearrange=no,resetcounter=no,suppressprefix]

           %Conversao dec -> bin e hex (2 alts) [6 pts]
           %Equivalencias (2 alts) [6 pts]
           %V ou F (3 alts) [6 pts]
	\begin{question}
      Os números abaixo estão em decimal. Efetue a conversão para binário e hexadecimal [6 pontos].
      
      
       	    \begin{tabular}{p{9cm}|p{10cm}}
	a) \word{{33}{35}{37}{39}}:		&b) \word{{126}{125}{124}{123}}: \\
					& \\
					& \\
					& \\
					& \\
					& \\
					& \\
					& \\
					& \\
					& \\
					& \\
					& \\
					& \\
					& \\
					& \\
					& \\
					& \\
					& \\
					& \\
					& \\
					& \\
					& \\
					& \\
	Valor em binário:\key{\word{100{001}{011}{101}{111 }}}& Valor em binário: \key{1111\word{{110}{101}{100}{011}}}\\
	Valor em hexadecimal:\key{\word{{21}{23}{25}{27}}}   & Valor em hexadecimal: \key{7\word{{E}{D}{C}{B}}}\\		
	\end{tabular}
    \end{question}
    \begin{question}
      Dado as proposições abaixo, determine se seguintes expressões lógicas são verdadeiras ou falsas [6 pontos]:
      
      \begin{multicols}{2}[]
	    \begin{itemize}
	    \item $p$: cachorros sabem voar
	    \item $q$: vacas têm 5 patas
	    \item $r$: 3+3=6
	    \item $s$: 5+5=10
	    \end{itemize}
      \end{multicols}
      
            	    \begin{tabular}{p{4cm}|p{5cm}|p{4cm}}
	a) $s \land q \lor r$:		&b) $[(p \land r) \lor s] \to s$: & c) $(p \lor r \land q) \leftrightarrow (r \land s)$\\
					& \key{$[(F \land V) \lor V] \to V$} &\key{$(F \lor V \land F) \leftrightarrow (V \land V)$}\\
	\key{$V \land F \lor V$}	& \key{$[F lor V] \to V$}	  &\key{$(F \lor F) \leftrightarrow (V \land V)$}\\
	\key{$F \lor V$}		& \key{$ V \to V$}		  &\key{$F \leftrightarrow V$}\\
	\key{$V$}			& \key{$V$}			  &\key{$F$}\\
					& 				  &\\
					& 				  &\\
					& 				  &\\
					& 				  &\\
					& 				  &\\
					& 				  &\\
					& 				  &\\
					& 				  &\\
					& 				  &\\
					& 				  &\\
					& 				  &\\
					& 				  &\\
					& 				  &\\
					& 				  &\\
					& 				  &\\
					& 				  & \\
					&				  &\\
					& 				  &\\			
	\end{tabular}
    \end{question}
    \begin{question}
	  Verifique as equivalências abaixo sem utilizar a tabela verdade [6 pontos]:
	         	    
	         	    
	         	    \begin{tabular}{p{9cm}|p{10cm}}
	a) $[(p \to q) \land (p \to r)] \equiv \neg (q \land r) \to \neg p$:	&b) $[\neg p \to (p \land q)] \land q \equiv p \land q$: \\
										& \\
					& \\
					& \\
					& \\
					& \\
					& \\
					& \\
					& \\
					& \\
					& \\
					& \\
					& \\
					& \\
					& \\
					& \\
					& \\
					& \\
					& \\
					& \\
					& \\
					& \\
	\end{tabular}
    \end{question}


\end{fillin}
\end{document}
