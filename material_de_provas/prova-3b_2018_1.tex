\documentclass[16pt]{examdesign}


\usepackage{multicol}
\usepackage{wrapfig}
\usepackage{graphics}
\usepackage{graphicx}
\usepackage{pgf,pgfarrows,pgfnodes,pgfautomata,pgfheaps,pgfshade}
\usepackage{amsmath}
\usepackage{pifont}
%\usepackage{gensymb}
\usepackage{textcomp}
\usepackage{multirow}

%\SectionFont{\large\sffamily}
\Fullpages
\ContinuousNumbering
%\ShortKey
%\NoKey

\DefineAnswerWrapper{}{}
\NumberOfVersions{4}
%\IncludeFromFile{foobar.tex}
\usepackage{colortbl}
\usepackage{amssymb}
\usepackage[english]{babel}
\usepackage[utf8]{inputenc}
\setrandomseed{1373}
\usepackage[portuguese]{algorithm2e}
\definecolor{lightgray}{rgb}{0.85, 0.85, 0.85}
\newtheorem{theorem}{THEOREM}
\newtheorem{proof}{PROOF}
\newcommand{\hgrayline}{\arrayrulecolor{lightgray}\hline\arrayrulecolor{black}}

\addtolength{\textwidth}{80pt}
\addtolength{\hoffset}{-55pt}


\addtolength{\voffset}{-30pt}
\addtolength{\textheight}{50pt}

%%%%%%%%%%%%%%%Imagens%%%%%%%%%%%%%%%%
\pgfdeclaremask{grafo1}{grafo1}
\pgfdeclareimage[mask=grafo1,height=3.5cm]{grafo1}{img/grafos_1.png}

\pgfdeclaremask{grafo2}{grafo2}
\pgfdeclareimage[mask=grafo2,height=3cm]{grafo2}{img/grafos_2.png}

\pgfdeclaremask{grafo}{grafo}
\pgfdeclareimage[mask=grafo,height=3cm]{grafo}{img/grafo.png}

\pgfdeclaremask{grafoPR}{grafoPR}
\pgfdeclareimage[mask=grafoPR, height=1.7cm]{grafoPR}{img/grafoPR.png}

\pgfdeclaremask{dijkstra}{dijkstra}
\pgfdeclareimage[mask=dijkstra, height=3cm]{dijkstra}{img/dijkstra.png}
%%%%%%%%%%%%%%%%%%%%%%%%%%%%%%%%%%%%%%%%%%%%%%%%%%%%%%%%%



\setlength{\fboxsep}{0pt}%
\setlength{\fboxrule}{1pt}

\newcommand{\aspas}{``}
\hyphenation{cor-res-pon-de}
\hyphenation{usa-do}
\hyphenation{tem-pe-ra-tu-ra}
% \newenvironment{Figure}
%   {\par\medskip\noindent\minipage{\linewidth}}
%   {\endminipage\par\medskip}

%\setlength\columnsep{25pt}

      \pgfdeclaremask{regras}{regras}
      \pgfdeclareimage[mask=regras,height=7cm]{regras}{img/regra_inferencia.png}
      
\begin{document}


\begin{examtop}

      \pgfdeclaremask{cefetLogo}{cefetLogo}
      \pgfdeclareimage[mask=cefetLogo,height=2cm]{cefetLogo}{../../../img/cefetLogo.png}
      
    \begin{multicols}{2}[]
    \begin{minipage}{2.5cm}
      \pgfuseimage{cefetLogo}
    \end{minipage}
      \columnbreak
      \begin{minipage}{400px}
      \begin{center}\large{\textbf{Matemática Discreta - Prova 3}}\end{center}~\\
      Professor: Daniel Hasan Dalip~~~~~~~~Valor: 33 Pontos~~~~~Belo Horizonte, 25/06/2018\\~\\
      Nome do(a) Aluno(a): \_\_\_\_\_\_\_\_\_\_\_\_\_\_\_\_\_\_\_\_\_\_\_\_	\_\_\_\_\_\_\_\_\_\_\_\_\_\_\_\_\_\_\_\_\_\_\_\_\_\_\_\_\_\_\_\_\_\_\_\_\_\_\_\_\_\_\_\_\_\_\_\_\_\_\_\_\_\_\_~
      Nota:\_\_\_\_\_\_\_\_\_\_\_\\       
      \end{minipage}



    \end{multicols}
  \hrulefill
  \begin{multicols}{2}[]
  
    \begin{itemize}
    \item Prova individual e sem consulta.\vspace{-6pt}
    \item Não é permitido o uso de aparelhos eletrônicos  \vspace{-6pt}
    \item A interpretação da questão faz parte da avaliação  \vspace{-6pt}
    \item A prova pode ser feita a lápis ou caneta  \vspace{-6pt}
    \item Qualquer tentativa de cola/fraude a prova será zerada  \vspace{-6pt}
    \end{itemize}
  \end{multicols}

\def\arraystretch{1}
\end{examtop}


\begin{multiplechoice}[title={Questões de múltipla escolha (3 pontos cada) },rearrange=no, resetcounter=no,suppressprefix,examcolumns=2,keycolumns=2]
   \begin{question}
   \vspace{-30pt}
      Considere o grafo representado pela seguinte matriz de adjacência: 

      \begin{center}
 


      \begin{tabular}{ccccc}
                       & A                      & B                      & C                      & D                      \\ \cline{2-5} 
\multicolumn{1}{c|}{A} & \multicolumn{1}{c|}{0} & \multicolumn{1}{c|}{1} & \multicolumn{1}{c|}{0} & \multicolumn{1}{c|}{1}\\ \cline{2-5} 
\multicolumn{1}{c|}{B} & \multicolumn{1}{c|}{0} & \multicolumn{1}{c|}{0} & \multicolumn{1}{c|}{1} & \multicolumn{1}{c|}{1}\\ \cline{2-5} 
\multicolumn{1}{c|}{C} & \multicolumn{1}{c|}{1} & \multicolumn{1}{c|}{0} & \multicolumn{1}{c|}{0} & \multicolumn{1}{c|}{1}\\ \cline{2-5} 
\multicolumn{1}{c|}{D} & \multicolumn{1}{c|}{0} & \multicolumn{1}{c|}{1} & \multicolumn{1}{c|}{0} & \multicolumn{1}{c|}{0}\\ \cline{2-5} 
\end{tabular}
\end{center}

Considerando o algoritmo HITS, inicializando a autoridade de cada vértice como o \textbf{grau de entrada} do mesmo, 
qual é o valor de Hub do vértice \textbf{\word{{C}{A}}} após a primeira iteração (sem normalizar)?
      \choice{1}
      \choice{2}
      \choice{3}
      \choice{4}
      \choice{5}
      \key{~\\~\\ Resposta: \word{{d}{e}}}
    \end{question}
      \begin{question}
	[Adaptado UFES, 2015] Um baralho é composto de 52
cartas, sendo que há 13 cartas de cada um dos 4 naipes
(paus, ouros, copas e espadas). A menor quantidade de
cartas que devem ser retiradas do baralho, sem olhar o
naipe, de modo a se garantir, com certeza, que sejam re-
tiradas pelo menos 2 cartas de um mesmo naipe é igual
a:
		\choice{3}
		\choice{4}
		\choice[!]{5}	
		\choice{13}
		\choice{14}

  \end{question}
 
  

    \begin{block}[questions=2]
  \columnbreak

  Para as questões \thefirst\ e \thelast, considere os seguintes grafos $G_1$, $G_2$, $G_3$ e $G_4$.
  
  
     \word{{\pgfuseimage{grafo2}}{\pgfuseimage{grafo1}}}
 

    \begin{question}
      Dentre os grafos, qual pode ser considerado um grafo completo? 
      
      \choice{$G_1$}
      \choice{$G_3$}
      \choice{$G_4$}
      \choice{$G_1$ e \word{{$G_4$}{$G_3$}}}
      \choice{$G_1$, $G_2$ e \word{{$G_3$}{$G_4$}}}
      \key{~\\~\\ Resposta: \word{{c} {b}}}
    \end{question}      
    \begin{question}
      Dentre os grafos, qual pode ser considerado um grafo conexo?
      \choice{$G_1$, $G_2$ e $G_3$}
      \choice{$G_1$, $G_2$ e $G_4$}
      \choice{$G_2$, $G_3$ e $G_4$}
      \choice{$G_2$ e $G_3$}
      \choice{$G_2$ e $G_4$}    
      \key{~\\~\\ Resposta: \word{{b} {a}}}
    \end{question}
        
  \end{block}
   \begin{question}
  [FDC, Fiocruz, 2014] Três técnicos e quatro engenheiros elaboraram um plano de melhorias das condições ambientais no trabalho. Dois técnicos e dois engenheiros serão escolhidos, dentre eles, para apresentar o plano à direção da empresa. O número de diferentes equipes de apresentação que podem ser formadas é igual a:

    \choice{$C(3,2) \word{{+}{\times}} C(4,2)$}
    \choice{$C(3,2) \word{{\times}{+}} C(4,2)$}
    \choice{$P(3,2) + P(4,2)$}
    \choice{$P(3,2) \times P(4,2)$}
    \choice{$P(7,4)$}
  
	
		
		%\choice{\word{{\FD}{\V}  {\FD}{\FD}}}
		\key{~\\~\\ Resposta: \word{{b} {a}}}
  \end{question}

\end{multiplechoice}
\begin{fillin}[title={},
                    rearrange=no,resetcounter=no,suppressprefix]

[5 pontos] 
          Responda as questões de forma mais clara possível.
          O aluno também perderá ponto caso não esteja clara a resposta ou se não foi mostrado como que se chegou em um determinado resultado.
  \begin{question}
	  Considere o seguinte grafo, representado pela matriz de adjacência:
	  
	  \begin{center}
	  \begin{tabular}{ccclc}
                       & A                      & B                      & C                      & D                      \\ \cline{2-5} 
\multicolumn{1}{c|}{A} & \multicolumn{1}{c|}{0} & \multicolumn{1}{c|}{1} & \multicolumn{1}{l|}{1} & \multicolumn{1}{c|}{0} \\ \cline{2-5} 
\multicolumn{1}{l|}{B} & \multicolumn{1}{l|}{0} & \multicolumn{1}{l|}{0} & \multicolumn{1}{l|}{0} & \multicolumn{1}{l|}{1} \\ \cline{2-5} 
\multicolumn{1}{c|}{C} & \multicolumn{1}{c|}{0} & \multicolumn{1}{c|}{1} & \multicolumn{1}{l|}{0} & \multicolumn{1}{c|}{1} \\ \cline{2-5} 
\multicolumn{1}{c|}{D} & \multicolumn{1}{c|}{1} & \multicolumn{1}{c|}{0} & \multicolumn{1}{l|}{1} & \multicolumn{1}{c|}{0} \\ \cline{2-5} 
\end{tabular}
\end{center}
	  
	  Calcule a primeira iteração do PageRank de cada nodo (dumping factor $d = 1$) utilize a equação do PageRank e considere que os nodos 
	  foram inicializados com os seguintes valores: $A=0.4; B=0.6; C=0.8; D=0.2$. Apresente como resultado o valor normalizado. 
	  Resposta sem a memória de cálculo não será considerada.\\
	  
      \begin{multicols}{2}[]
	    \begin{enumerate}
	    \item[a)] Valor para o vértice $A$: \blank{0,1~~~~~~~~~} 
	    \item[b)] Valor para o vértice $B$: \blank{0,3~~~~~~~~~} 
	    \item[c)] Valor para o vértice $C$: \blank{0,1~~~~~~~~~} 
	    \item[d)] Valor para o vértice $D$: \blank{0,5~~~~~~~~~} 
	    \end{enumerate}
      \end{multicols}

  \pagebreak
  \end{question}          
          \begin{question}
		[6 pontos] Considere o grafo: 
\begin{center}
\pgfuseimage{dijkstra}
\end{center}

Execute o algoritmo de Dijsktra para o grafo a seguir tendo B como vértice de origem. 
Apresente nas tabelas abaixo, após o final de cada iteração  (i.e. iteração referente ao while mais externo, ou seja, após percorrer todos os vértices adjacentes do vértice extraído u): o vértice extraído u, 
a fila de prioridade Q  os vetores p (representando o caminho) e d (representando a distância do vértice de origem até o atual). 
\begin{multicols}{4}
  \begin{tabular}{|llllll|}
  \hline
  \multicolumn{6}{|c|}{Iteração \# 1}                                                                                                          \\
					    &                       &                       &                       &                       &  \\
  u:                                        &                       &                       &                       &                       &  \\ \cline{2-3}
					    &                       &                       &                       &                       &  \\ \cline{2-5}
  \multicolumn{1}{|r|}{Q:}                  & \multicolumn{1}{l|}{} & \multicolumn{1}{l|}{} & \multicolumn{1}{l|}{} & \multicolumn{1}{l|}{} &  \\
  \multicolumn{1}{|l|}{}                    & \multicolumn{1}{l|}{} & \multicolumn{1}{l|}{} & \multicolumn{1}{l|}{} & \multicolumn{1}{l|}{} &  \\ \cline{2-5}
					    &                       &                       &                       &                       &  \\ \cline{2-5}
  \multicolumn{1}{|r|}{\multirow{2}{*}{p:}} & \multicolumn{1}{l|}{} & \multicolumn{1}{l|}{} & \multicolumn{1}{l|}{} & \multicolumn{1}{l|}{} &  \\
  \multicolumn{1}{|r|}{}                    & \multicolumn{1}{l|}{} & \multicolumn{1}{l|}{} & \multicolumn{1}{l|}{} & \multicolumn{1}{l|}{} &  \\ \cline{2-5}
					    & A                     & B                     & C                     & D                     &  \\
					    &                       &                       &                       &                       &  \\ \cline{2-5}
  \multicolumn{1}{|l|}{\multirow{2}{*}{d:}} & \multicolumn{1}{l|}{} & \multicolumn{1}{l|}{} & \multicolumn{1}{l|}{} & \multicolumn{1}{l|}{} &  \\
  \multicolumn{1}{|l|}{}                    & \multicolumn{1}{l|}{} & \multicolumn{1}{l|}{} & \multicolumn{1}{l|}{} & \multicolumn{1}{l|}{} &  \\ \cline{2-5}
					    & A                     & B                     & C                     & D                     &  \\ \hline
  \end{tabular}
  \begin{tabular}{|llllll|}
  \hline
  \multicolumn{6}{|c|}{Iteração \# 2}                                                                                                          \\
					    &                       &                       &                       &                       &  \\
  u:                                        &                       &                       &                       &                       &  \\ \cline{2-3}
					    &                       &                       &                       &                       &  \\ \cline{2-5}
  \multicolumn{1}{|r|}{Q:}                  & \multicolumn{1}{l|}{} & \multicolumn{1}{l|}{} & \multicolumn{1}{l|}{} & \multicolumn{1}{l|}{} &  \\
  \multicolumn{1}{|l|}{}                    & \multicolumn{1}{l|}{} & \multicolumn{1}{l|}{} & \multicolumn{1}{l|}{} & \multicolumn{1}{l|}{} &  \\ \cline{2-5}
					    &                       &                       &                       &                       &  \\ \cline{2-5}
  \multicolumn{1}{|r|}{\multirow{2}{*}{p:}} & \multicolumn{1}{l|}{} & \multicolumn{1}{l|}{} & \multicolumn{1}{l|}{} & \multicolumn{1}{l|}{} &  \\
  \multicolumn{1}{|r|}{}                    & \multicolumn{1}{l|}{} & \multicolumn{1}{l|}{} & \multicolumn{1}{l|}{} & \multicolumn{1}{l|}{} &  \\ \cline{2-5}
					    & A                     & B                     & C                     & D                     &  \\
					    &                       &                       &                       &                       &  \\ \cline{2-5}
  \multicolumn{1}{|l|}{\multirow{2}{*}{d:}} & \multicolumn{1}{l|}{} & \multicolumn{1}{l|}{} & \multicolumn{1}{l|}{} & \multicolumn{1}{l|}{} &  \\
  \multicolumn{1}{|l|}{}                    & \multicolumn{1}{l|}{} & \multicolumn{1}{l|}{} & \multicolumn{1}{l|}{} & \multicolumn{1}{l|}{} &  \\ \cline{2-5}
					    & A                     & B                     & C                     & D                     &  \\ \hline
  \end{tabular}
  \begin{tabular}{|llllll|}
  \hline
  \multicolumn{6}{|c|}{Iteração \# 3}                                                                                                          \\
					    &                       &                       &                       &                       &  \\
  u:                                        &                       &                       &                       &                       &  \\ \cline{2-3}
					    &                       &                       &                       &                       &  \\ \cline{2-5}
  \multicolumn{1}{|r|}{Q:}                  & \multicolumn{1}{l|}{} & \multicolumn{1}{l|}{} & \multicolumn{1}{l|}{} & \multicolumn{1}{l|}{} &  \\
  \multicolumn{1}{|l|}{}                    & \multicolumn{1}{l|}{} & \multicolumn{1}{l|}{} & \multicolumn{1}{l|}{} & \multicolumn{1}{l|}{} &  \\ \cline{2-5}
					    &                       &                       &                       &                       &  \\ \cline{2-5}
  \multicolumn{1}{|r|}{\multirow{2}{*}{p:}} & \multicolumn{1}{l|}{} & \multicolumn{1}{l|}{} & \multicolumn{1}{l|}{} & \multicolumn{1}{l|}{} &  \\
  \multicolumn{1}{|r|}{}                    & \multicolumn{1}{l|}{} & \multicolumn{1}{l|}{} & \multicolumn{1}{l|}{} & \multicolumn{1}{l|}{} &  \\ \cline{2-5}
					    & A                     & B                     & C                     & D                     &  \\
					    &                       &                       &                       &                       &  \\ \cline{2-5}
  \multicolumn{1}{|l|}{\multirow{2}{*}{d:}} & \multicolumn{1}{l|}{} & \multicolumn{1}{l|}{} & \multicolumn{1}{l|}{} & \multicolumn{1}{l|}{} &  \\
  \multicolumn{1}{|l|}{}                    & \multicolumn{1}{l|}{} & \multicolumn{1}{l|}{} & \multicolumn{1}{l|}{} & \multicolumn{1}{l|}{} &  \\ \cline{2-5}
					    & A                     & B                     & C                     & D                     &  \\ \hline
  \end{tabular}
  \begin{tabular}{|llllll|}
  \hline
  \multicolumn{6}{|c|}{Iteração \# 4}                                                                                                          \\
					    &                       &                       &                       &                       &  \\
  u:                                        &                       &                       &                       &                       &  \\ \cline{2-3}
					    &                       &                       &                       &                       &  \\ \cline{2-5}
  \multicolumn{1}{|r|}{Q:}                  & \multicolumn{1}{l|}{} & \multicolumn{1}{l|}{} & \multicolumn{1}{l|}{} & \multicolumn{1}{l|}{} &  \\
  \multicolumn{1}{|l|}{}                    & \multicolumn{1}{l|}{} & \multicolumn{1}{l|}{} & \multicolumn{1}{l|}{} & \multicolumn{1}{l|}{} &  \\ \cline{2-5}
					    &                       &                       &                       &                       &  \\ \cline{2-5}
  \multicolumn{1}{|r|}{\multirow{2}{*}{p:}} & \multicolumn{1}{l|}{} & \multicolumn{1}{l|}{} & \multicolumn{1}{l|}{} & \multicolumn{1}{l|}{} &  \\
  \multicolumn{1}{|r|}{}                    & \multicolumn{1}{l|}{} & \multicolumn{1}{l|}{} & \multicolumn{1}{l|}{} & \multicolumn{1}{l|}{} &  \\ \cline{2-5}
					    & A                     & B                     & C                     & D                     &  \\
					    &                       &                       &                       &                       &  \\ \cline{2-5}
  \multicolumn{1}{|l|}{\multirow{2}{*}{d:}} & \multicolumn{1}{l|}{} & \multicolumn{1}{l|}{} & \multicolumn{1}{l|}{} & \multicolumn{1}{l|}{} &  \\
  \multicolumn{1}{|l|}{}                    & \multicolumn{1}{l|}{} & \multicolumn{1}{l|}{} & \multicolumn{1}{l|}{} & \multicolumn{1}{l|}{} &  \\ \cline{2-5}
					    & A                     & B                     & C                     & D                     &  \\ \hline
  \end{tabular}
\end{multicols}


\pagebreak
\end{question}
     \begin{question}
    
[6 pontos] Na rotina de estudos de João, ele deverá estudar Matemática 1, Matemática 2, Português 1, Português 2, Programação 1 e Demonstrações de Teoremas. 
Para estudar Matemática 2, deve-se estudar primeiramente Matemática 1. Para estudar Português 2, deve-se estudar primeiramente Português 1.
Além disso, a disciplina Demonstrações de Teoremas deve ser estudada apenas após estudar Matemática \word{{2}{1}} e Português \word{{1}{2}}. Considere que deseja-se resolver o problema usando o algoritmo de ordenação topológica. [6 pontos]

	  \begin{itemize}
		  \item[(a)] Para utilizar tal algoritmo, o que o vértice irá representar?\\~\\ \blank{~~~~~~~~~~~~~~~~~~~~~~~~~~~~~~~~~~~~~~~~~~~~~~~~~~~~~~~~~~~~~~~~~~~~~~~~~~~~~~~~~~~~~~~~~~~~~~~~~~~~~~~~~~~~~~~~~~~~~~~~~~~~~~~~~~~~~~~~~~~~~~~~~~~~~~~~}
		  \item[(b)] Neste algoritmo, o que a aresta irá representar?\\~\\
		  \blank{~~~~~~~~~~~~~~~~~~~~~~~~~~~~~~~~~~~~~~~~~~~~~~~~~~~~~~~~~~~~~~~~~~~~~~~~~~~~~~~~~~~~~~~~~~~~~~~~~~~~~~~~~~~~~~~~~~~~~~~~~~~~~~~~~~~~~~~~~~~~~~~~~~~~~~~~}
		  \item[(c)] Desenhe o grafo correspondente. Dica: para facilitar, use siglas: P1, P2, M1, M2, DT correspondendo, respectivamente, à Português 1, Português 2, Matemática 1, 
		  Matemática 2 e Demonstrações de Teoremas.
		  ~\\
		  ~\\
		  ~\\
		  ~\\
		  ~\\
		  ~\\
		  ~\\
		  ~\\
		  ~\\
		  ~\\		  ~\\
		  ~\\
		  ~\\
		  ~\\		  ~\\
		  ~\\
		  ~\\
		  ~\\
		  \item[(d)] O algoritmo de ordenação topológica possui uma lista para armazenar o resultado deste problema. 
		  Simule a execuçao deste algoritmo mostrando, passo a passo, o preenchimento dessa lista (cada vez que ela for atualizada). Comece a busca por Matemática 1 ou Português 1. 
		  Durante o caminhamento, caso seja necessário decidir qual vértice adjacente será visitado, visite sempre em ordem alfabética. Use as mesmas siglas da letra (c) para representar cada vértice.
		  
~\\
		  ~\\
		  ~\\
		  ~\\
		  ~\\
		  ~\\
		  ~\\
		  ~\\
		  ~\\
		  ~\\~\\
		  ~\\
		  ~\\
		  ~\\
		  ~\\
		  ~\\
		  ~\\
		  ~\\
		  ~\\
		  ~\\

	  \end{itemize}
    \end{question}  
    
\end{fillin}
\end{document}
